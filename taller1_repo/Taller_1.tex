\documentclass[12pt, letterpaper]{article}
\usepackage[utf8]{inputenc}
\usepackage[T1]{fontenc}
\usepackage{graphicx}
\usepackage{listings}
\usepackage{xcolor}
\usepackage[spanish, es-tabla]{babel} % es-tabla para que diga "Tabla" y no "Cuadro"
\addto\captionsspanish{\renewcommand{\contentsname}{Lista de contenido}}
\usepackage{geometry}
\usepackage{float}
\usepackage{hyperref}
\usepackage{titlesec} % Para mejorar el formato de los títulos
\usepackage{fancyhdr} % Para encabezados y pies de página
\usepackage{tikz}
\usepackage{atbegshi} % Para controlar en qué páginas aparece el marco
\usepackage{eso-pic} % NECESARIO para \AddToShipoutPictureBG
\usetikzlibrary{calc} % Recomendado para cálculos de coordenadas en el marco
\definecolor{VerdeMilitar}{RGB}{75, 83, 32} % Definición del color Verde Militar (RGB)

% Configuración de márgenes
\geometry{
    left=2.5cm,
    right=2.5cm,
    top=2.5cm,
    bottom=2.5cm
}

% Quitar bordes rojos de hipervínculos
\hypersetup{
    colorlinks=true,
    linkcolor=black,
    urlcolor=blue,
    citecolor=black
}

% --- COLORES Y ESTILO DE CÓDIGO (Para comandos de terminal/Cisco) ---
\definecolor{termback}{rgb}{0.95,0.95,0.95}
\lstset{
    backgroundcolor=\color{termback},
    basicstyle=\ttfamily\small,
    breaklines=true,
    frame=single,
    captionpos=b
}

% --- DATOS DEL TALLER (Modifica esto en cada entrega) ---
\newcommand{\tituloTaller}{Conocer un Centro de Datos}
\newcommand{\unidadNumero}{1}
\newcommand{\semanaNumero}{1}
\newcommand{\codigoActividad}{TS1} % Taller Semana 1
\newcommand{\equipoNumero}{05}      % Tu número de equipo
\newcommand{\integranteA}{Moya Monreal Erick Anselmo --- 1110604}
\newcommand{\integranteB}{Rodríguez Maldonado Irving Alejandro --- 1182794}
% -------------------------------------------------------

% Configuración del Marco Elegante
\newcommand{\LogosMarco}{
    \begin{tikzpicture}[remember picture, overlay]
        % Marco Externo (Verde Militar)
        \draw [line width=1.2pt, color=VerdeMilitar] 
            ($(current page.north west) + (0.7cm,-0.7cm)$) 
            rectangle 
            ($(current page.south east) + (-0.7cm,0.7cm)$);
            
        % Marco Interno (Verde Militar más claro/tenue)
        \draw [line width=0.6pt, color=VerdeMilitar!60] 
            ($(current page.north west) + (0.85cm,-0.85cm)$) 
            rectangle 
            ($(current page.south east) + (-0.85cm,0.85cm)$);
    \end{tikzpicture}
}

% --- CONFIGURACIÓN DE ENCABEZADO ---
\pagestyle{fancy}
\fancyhf{}
\lhead{\footnotesize Redes de Computadoras}
\chead{\footnotesize \codigoActividad: \tituloTaller}
\rhead{\footnotesize Equipo \equipoNumero}
\cfoot{\thepage}

% Cambiar color de la línea del encabezado
\renewcommand{\headrule}{%
    {\color{VerdeMilitar}\hrule width\headwidth height 0.5pt \vskip-\headrulewidth}
}

\begin{document}
    
    % --- PORTADA ---
    \begin{titlepage}
        \begin{center}
            \includegraphics[width=4cm]{logo.PNG}
            \hspace{1cm}
            \includegraphics[width=5cm]{1593484343817.jpg}
            \vspace{1.5cm}

            \LARGE \textbf{Universidad Autónoma de Baja California}\\
            \Large Facultad de Ingeniería Mexicali\\
            \Large Ingeniería en Computación\\
            \vspace{0.7cm}

            {\color{VerdeMilitar}\rule{\linewidth}{0.5mm}} \\
            \large \textbf{Reporte de Actividad:}\\
            \vspace{0.1cm}
            {\Huge {\tituloTaller} }\\
            {\color{VerdeMilitar}\rule{\linewidth}{0.5mm}}

            \vspace{0.7cm}
            \begin{minipage}{0.45\textwidth}
                \textbf{Materia:}\\ Redes de Computadoras\\
                \textbf{Unidad:} \unidadNumero \quad \textbf{Semana:} \semanaNumero
            \end{minipage}
            \begin{minipage}{0.45\textwidth}
                \flushright
                \textbf{Profesor:}\\ Mto. Emmanuel Zúñiga Torres
            \end{minipage}
      
            \vfill           
            \textbf{EQUIPO \equipoNumero :}\\
            \vspace{0.3cm}
            {\color{VerdeMilitar}\rule{4cm}{0.1mm}}\\
            \integranteA \\
            \integranteB \\
            
            \vspace{1.5cm}
            
            \vfill
            
            Mexicali, Baja California, \today
        \end{center}
    \end{titlepage}
    
    \newpage

    % Activamos el marco justo aquí
    \AddToShipoutPictureBG{\LogosMarco}

    \tableofcontents % Recuerda compilar dos veces
    \newpage
    
    % --- SECCIÓN 1: INTRODUCCIÓN ---
    \section{Introducción}
    Este taller se enfoca en el análisis de la infraestructura de los centros de datos modernos...

    [Image of data center network architecture diagram]


    \section{Desarrollo del Taller}
    
    \subsection{Infraestructura y Capas de Seguridad}
    Durante la visualización de los recursos de Google, se identificaron los siguientes puntos críticos de la red:
    
    \begin{itemize}
        \item \textbf{Capa Física:} Servidores personalizados y racks de alta densidad.
        \item \textbf{Capa de Enlace:} Switch Fabric y conectividad mediante fibra óptica.
    \end{itemize}

    % Ejemplo de cómo insertar una imagen de evidencia
    \begin{figure}[H]
        \centering
        %\includegraphics[width=0.7\textwidth]{evidencia1.png} 
        \caption{Captura de pantalla: Observación de las capas de seguridad física.}
    \end{figure}

    \subsection{Análisis Técnico}
    \begin{lstlisting}[caption=Ejemplo de verificación de red (Opcional)]
    ping 8.8.8.8
    tracert google.com
    \end{lstlisting}

    \begin{table}[H]
    \centering
    \begin{tabular}{|l|p{10cm}|}
    \hline
    \textbf{Capa} & \textbf{Descripción Técnica} \\ \hline
    1. Perímetro & Vallas, señalización y patrullas externas. \\ \hline
    2. Patio Seguro & Barreras vehiculares y cámaras térmicas. \\ \hline
    3. Acceso & Controles biométricos (iris/huella) y trampas de personal. \\ \hline
    4. SOC & Monitoreo 24/7 y gestión de alarmas. \\ \hline
    5. Piso de Datos & Acceso restringido a técnicos autorizados. \\ \hline
    6. Destrucción & Trituración física de medios de almacenamiento. \\ \hline
    \end{tabular}
    \caption{Capas de seguridad física observadas en el recorrido virtual.}
    \end{table}

    \section{Conclusiones}
    \textbf{Individual - Erick Moya:} El taller me permitió entender que la seguridad física es el primer paso de la ciberseguridad en redes...
    
    \vspace{0.5cm}
    \textbf{Individual - Irving Rodríguez:} La complejidad de un centro de datos radica en su capacidad de escalabilidad...

    \begin{thebibliography}{9}
        \bibitem{ref1} Google Data Centers. (2026). Recuperado de \url{https://datacenters.google/}
    \end{thebibliography}

\end{document}