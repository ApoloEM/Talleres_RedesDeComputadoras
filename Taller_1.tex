\documentclass[12pt, letterpaper]{article}
\usepackage[utf8]{inputenc}
\usepackage[T1]{fontenc}
\usepackage{graphicx}
\usepackage{listings}
\usepackage{xcolor}
\usepackage[spanish]{babel}
\usepackage{geometry}
\usepackage{float}
\usepackage{hyperref}
\usepackage{titlesec} % Para mejorar el formato de los títulos

% Configuración de márgenes
\geometry{
    left=2.5cm,
    right=2.5cm,
    top=2.5cm,
    bottom=2.5cm
}

% Quitar bordes rojos de hipervínculos
\hypersetup{
    colorlinks=true,
    linkcolor=black,
    urlcolor=blue,
    citecolor=black
}

% --- DATOS DEL TALLER (Modifica esto en cada entrega) ---
\newcommand{\tituloTaller}{Tema del Taller}
\newcommand{\unidadNumero}{Unidad \#1 }
\newcommand{\semanaNumero}{Semana \#1 }
\newcommand{\codigoActividad}{TS\#1} % Taller Semana 1
\newcommand{\equipoNumero}{05}      % Tu número de equipo
% -------------------------------------------------------

\begin{document}
    
    % --- PORTADA ---
    \begin{titlepage}
        \begin{center}
            \includegraphics[width=4cm]{logo.PNG}
            \hspace{1cm}
            \includegraphics[width=5cm]{1593484343817.jpg}
            \vspace{1.5cm}

            \LARGE \textbf{Universidad Autónoma de Baja California}\\
            \Large Facultad de Ingeniería Mexicali\\
            \Large Ingeniería en Computación\\
            \vspace{2cm}

            \large \textbf{Reporte de Actividad:}\\
            \Huge \tituloTaller\\
            \vspace{0.5cm}
            \large \unidadNumero - \semanaNumero - \codigoActividad\\
            
            \vspace{0.3cm}
            \large \textbf{Redes de Computadoras}\\
            \vspace{0.3cm}
            \textbf{Profesor:}\\
            Mto. Emmanuel Zúñiga Torres
            \vspace{1.5cm}

            
            
            \vfill
            
            \textbf{Equipo \equipoNumero :}\\
            \vspace{0.3cm}
            Moya Monreal Erick Anselmo 1110604\\
            Rodríguez Maldonado Irving Alejandro 1182794\\
            
            \vspace{1.5cm}
            
            \vfill
            
            Mexicali, Baja California, \today
        \end{center}
    \end{titlepage}
    
    \newpage
    \tableofcontents % Recuerda compilar dos veces
    \newpage
    
    % --- SECCIÓN 1: INTRODUCCIÓN ---
    \section{Introducción}
    Se coloca la introducción correspondiente al tema del taller.

    % --- SECCIÓN 2: DESARROLLO / ACTIVIDAD ---
    \section{Desarrollo del Taller}
    
    \subsection{Subseccion 1}
    Dividir los temas vistos y correspondientes al taller
    
    \begin{itemize}
        \item \textbf{usar listas, tablas, recuerdos audiovisuales} Listar.
        \item \textbf{otro} Otro concepto.
    \end{itemize}    

    \subsection{Subseccion 2}
    Otra subseccion del tema.
    
    % --- SECCIÓN 3: CONCLUSIONES ---
    \section{Conclusiones}
    Se redactan Conclusiones. Preguntar si son individuales o en equipo.

    % --- SECCIÓN 4: REFERENCIAS ---
    \begin{thebibliography}{9}

    \end{thebibliography}

\end{document}