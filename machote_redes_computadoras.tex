\documentclass[12pt, letterpaper]{article}

% --- IDIOMA Y FUENTES ---
\usepackage[utf8]{inputenc}
\usepackage[T1]{fontenc}
\usepackage[spanish, es-tabla]{babel}

% --- FORMATO GENERAL ---
\usepackage{geometry}
% Configuración de márgenes
\geometry{
    left=2.5cm,
    right=2.5cm,
    top=2.5cm,
    bottom=2.5cm
}

\usepackage{graphicx}
\usepackage{float}
\usepackage{enumitem}
\usepackage{listings}
\usepackage{xcolor}
\usepackage{titlesec}
\usepackage{tikz}
\usepackage{atbegshi} 
\usepackage{eso-pic}
\usetikzlibrary{calc}

%--- DEFINICIÓN DE COLORES ---
\definecolor{VerdeMilitar}{RGB}{75, 83, 32}

% --- COLORES Y ESTILO DE CÓDIGO ---
\definecolor{termback}{rgb}{0.95,0.95,0.95}
\lstset{
    backgroundcolor=\color{termback},
    basicstyle=\ttfamily\small,
    breaklines=true,
    frame=single,
    captionpos=b
}

% --- CONFIGURACIÓN DE HYPERREF ---
\usepackage{hyperref}
\hypersetup{ % Quitar bordes rojos de hipervínculos
    colorlinks=true,
    linkcolor=black,
    urlcolor=blue,
    citecolor=black
}

% --- ENCABEZADO Y PIE ---
\usepackage{fancyhdr}
\pagestyle{fancy}
\fancyhf{}
\lhead{\footnotesize Redes de Computadoras}
\chead{\footnotesize \codigoActividad: \tituloTaller}
\rhead{\footnotesize Equipo \equipoNumero}
\cfoot{\thepage}

\renewcommand{\headrule}{%
    {\color{VerdeMilitar}\hrule width\headwidth height 0.5pt \vskip-\headrulewidth}
}

% --- TABLA DE CONTENIDO ---
\addto\captionsspanish{\renewcommand{\contentsname}{Lista de contenido}}

% --- DATOS DEL TRABAJO ---
\newcommand{\tituloTaller}{Nombre del Taller}
\newcommand{\unidadNumero}{X}
\newcommand{\semanaNumero}{X}
\newcommand{\codigoActividad}{TSX}
\newcommand{\equipoNumero}{05}

% --- INTEGRANTES ---
\newcommand{\integranteA}{Moya Monreal Erick Anselmo --- 1110604}
\newcommand{\integranteB}{Rodríguez Maldonado Irving Alejandro --- 1182794}
% Agrega más si es necesario

% Marco Elegante
\newcommand{\LogosMarco}{
    \begin{tikzpicture}[remember picture, overlay]
        \draw [line width=1.2pt, color=VerdeMilitar!90] 
            ($(current page.north west) + (0.7cm,-0.7cm)$) 
            rectangle 
            ($(current page.south east) + (-0.7cm,0.7cm)$);
        \draw [line width=0.6pt, color=VerdeMilitar!60] 
            ($(current page.north west) + (0.85cm,-0.85cm)$) 
            rectangle 
            ($(current page.south east) + (-0.85cm,0.85cm)$);
    \end{tikzpicture}
}

%==================================================================

\begin{document}

% --- PORTADA ---
\begin{titlepage}
    \begin{center}
    \includegraphics[width=4cm]{logo.png}
    \hspace{1cm}
    \includegraphics[width=5cm]{1593484343817.jpg}
    \vspace{1.5cm}

    \LARGE \textbf{Universidad Autónoma de Baja California}\\
    \Large Facultad de Ingeniería Mexicali\\
    \Large Ingeniería en Computación\\
    \vspace{0.7cm}

    {\color{VerdeMilitar}\rule{\linewidth}{0.5mm}} \\
    \large \textbf{Reporte de Actividad:}\\
    \vspace{0.1cm}
    {\Huge {\tituloTaller} }\\
    {\color{VerdeMilitar}\rule{\linewidth}{0.5mm}}

    \vspace{0.7cm}
    \begin{minipage}{0.45\textwidth}
        \textbf{Materia:}\\ Redes de Computadoras\\
        \textbf{Unidad:} \unidadNumero \quad \textbf{Semana:} \semanaNumero
    \end{minipage}
    \begin{minipage}{0.45\textwidth}
        \flushright
        \textbf{Profesor:}\\ Mtro. Emmanuel Zúñiga Torres
    \end{minipage}
      
    \vfill           
    \textbf{EQUIPO \equipoNumero :}\\
    \vspace{0.3cm}
    {\color{VerdeMilitar}\rule{4cm}{0.1mm}}\\
    \integranteA \\
    \integranteB \\
            
    \vspace{1.5cm}
    \vfill
    Mexicali, Baja California\\ \today
    \end{center}
\end{titlepage}

\newpage
\AddToShipoutPictureBG{\LogosMarco}
\tableofcontents

% --- SECCIÓN 1: INTRODUCCIÓN ---
\newpage
\section{Introducción}
Aquí se presenta el contexto del taller, los recursos utilizados y el objetivo
general de la actividad. Este apartado debe permitir al lector comprender
qué se hizo y con qué propósito.

% --- SECCIÓN 2: DESARROLLO DEL TALLER ---
\newpage
\section{Desarrollo de la Actividad}

\subsection{Tema o Apartado 1}
Desarrollo del contenido principal del taller.

\subsection{Tema o Apartado 2}
Explicaciones técnicas, análisis, observaciones, etc.

\begin{figure}[H]
    \centering
    \includegraphics[width=0.7\textwidth]{taller1/google_servers.png}
    \caption{Descripción de la imagen. Imagen de prueba.}
\end{figure}

% --- SECCIÓN 3: CONCLUSIONES ---
\newpage
\section{Conclusiones}
Conclusión general o conclusiones individuales sobre lo aprendido,
los aspectos más relevantes del taller y su importancia en el contexto
de la materia.

\newpage
\addcontentsline{toc}{section}{Referencias}
\begin{thebibliography}{9}

\bibitem{ref1}
Autor o Institución. \textit{Título del recurso}.
Recuperado de: \url{https://...}

\bibitem{ref2}
Autor o Institución. \textit{Título del recurso}.
Recuperado de: \url{https://...}

\end{thebibliography}

%             Y en el texto se citan así:
%... como se muestra en diversos centros de datos \cite{ref1}.
%==================================================================

\end{document}