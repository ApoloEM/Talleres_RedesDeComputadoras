\documentclass[12pt, letterpaper]{article}
\usepackage[utf8]{inputenc}
\usepackage[T1]{fontenc}
\usepackage{enumitem}
\usepackage{graphicx}
\usepackage{listings}
\usepackage{xcolor}
\usepackage[spanish, es-tabla]{babel} 
\addto\captionsspanish{\renewcommand{\contentsname}{Lista de contenido}}
\usepackage{geometry}
\usepackage{float}
\usepackage{hyperref}
\usepackage{titlesec}
\usepackage{fancyhdr}
\usepackage{tikz}
\usepackage{atbegshi}
\usepackage{eso-pic}
\usetikzlibrary{calc}
\definecolor{VerdeMilitar}{RGB}{75, 83, 32}

% Configuración de márgenes
\geometry{
    left=2.5cm,
    right=2.5cm,
    top=2.5cm,
    bottom=2.5cm
}

% Ajuste de altura del encabezado
\setlength{\headheight}{12.1pt}

% Quitar bordes rojos de hipervínculos
\hypersetup{
    colorlinks=true,
    linkcolor=black,
    urlcolor=blue,
    citecolor=black
}

% --- COLORES Y ESTILO DE CÓDIGO ---
\definecolor{termback}{rgb}{0.95,0.95,0.95}
\lstset{
    backgroundcolor=\color{termback},
    basicstyle=\ttfamily\small,
    breaklines=true,
    frame=single,
    captionpos=b
}

% --- DATOS DEL TALLER ---
\newcommand{\tituloTaller}{Nombre del Taller 2}
\newcommand{\unidadNumero}{X}
\newcommand{\semanaNumero}{X}
\newcommand{\codigoActividad}{TS2} 
\newcommand{\equipoNumero}{05}      
\newcommand{\integranteA}{Moya Monreal Erick Anselmo --- 1110604}
\newcommand{\integranteB}{Rodríguez Maldonado Irving Alejandro --- 1182794}

% Marco Elegante
\newcommand{\LogosMarco}{
    \begin{tikzpicture}[remember picture, overlay]
        \draw [line width=1.2pt, color=VerdeMilitar!90] 
            ($(current page.north west) + (0.7cm,-0.7cm)$) 
            rectangle 
            ($(current page.south east) + (-0.7cm,0.7cm)$);
        \draw [line width=0.6pt, color=VerdeMilitar!60] 
            ($(current page.north west) + (0.85cm,-0.85cm)$) 
            rectangle 
            ($(current page.south east) + (-0.85cm,0.85cm)$);
    \end{tikzpicture}
}

% --- ENCABEZADO ---
\pagestyle{fancy}
\fancyhf{}
\lhead{\footnotesize Redes de Computadoras}
\chead{\footnotesize \codigoActividad: \tituloTaller}
\rhead{\footnotesize Equipo \equipoNumero}
\cfoot{\thepage}

\renewcommand{\headrule}{%
    {\color{VerdeMilitar}\hrule width\headwidth height 0.5pt \vskip-\headrulewidth}
}

\begin{document}

% --- PORTADA ---
\begin{titlepage}
	\begin{center}
		\includegraphics[width=4cm]{../logo.png}
		\hspace{1cm}
		\includegraphics[width=5cm]{../1593484343817.jpg}
		\vspace{1.5cm}

		\LARGE \textbf{Universidad Autónoma de Baja California}\\
		\Large Facultad de Ingeniería Mexicali\\
		\Large Ingeniería en Computación\\
		\vspace{0.7cm}

		{\color{VerdeMilitar}\rule{\linewidth}{0.5mm}} \\
		\large \textbf{Reporte de Actividad:}\\
		\vspace{0.1cm}
		{\Huge {\tituloTaller} }\\
		{\color{VerdeMilitar}\rule{\linewidth}{0.5mm}}

		\vspace{0.7cm}
		\begin{minipage}{0.45\textwidth}
			\textbf{Materia:}\\ Redes de Computadoras\\
			\textbf{Unidad:} \unidadNumero \quad \textbf{Semana:} \semanaNumero
		\end{minipage}
		\begin{minipage}{0.45\textwidth}
			\flushright
			\textbf{Profesor:}\\ Mtro. Emmanuel Zúñiga Torres
		\end{minipage}

		\vfill
		\textbf{EQUIPO \equipoNumero :}\\
		\vspace{0.3cm}
		{\color{VerdeMilitar}\rule{4cm}{0.1mm}}\\
		\integranteA \\
		\integranteB \\

		\vspace{1.5cm}
		\vfill
		Mexicali, Baja California\\ \today
	\end{center}
\end{titlepage}

\newpage
\AddToShipoutPictureBG{\LogosMarco}
\tableofcontents
\newpage

% --- SECCIÓN 1: INTRODUCCIÓN ---
\section{Introducción}

Aquí se presenta el contexto del taller, los recursos utilizados y el objetivo general de la actividad. Este apartado debe permitir al lector comprender qué se hizo y con qué propósito.

\newpage
% --- SECCIÓN 2: DESARROLLO DEL TALLER ---
\section{Desarrollo}

\subsection{Tema o Apartado 1}
Desarrollo del contenido principal del taller.

\subsection{Tema o Apartado 2}
Explicaciones técnicas, análisis, observaciones, etc.

% Ejemplo de figura (descomenta y ajusta la ruta según sea necesario)
%\begin{figure}[H]
%    \centering
%    \fbox{\includegraphics[width=0.7\textwidth]{nombre_de_imagen.png}}
%    \caption{Descripción de la imagen.}
%\end{figure}

\newpage
\section{Conclusiones}

Conclusión general o conclusiones individuales sobre lo aprendido, los aspectos más relevantes del taller y su importancia en el contexto de la materia.

\newpage
\addcontentsline{toc}{section}{Referencias}
\begin{thebibliography}{9}

	\bibitem{ref1}
	Autor o Institución. \textit{Título del recurso}.
	Recuperado de: \url{https://...}

	\bibitem{ref2}
	Autor o Institución. \textit{Título del recurso}.
	Recuperado de: \url{https://...}

\end{thebibliography}

\end{document}
