%==================================================================
% CLASE DEL DOCUMENTO
% Define el tipo de documento, tamaño de letra y papel
% NO MODIFICAR a menos que se sepa lo que se hace
%==================================================================
\documentclass[12pt, letterpaper]{article}

%==================================================================
% IDIOMA, CODIFICACIÓN Y FUENTES
% Permiten escribir acentos, ñ y reglas correctas en español
% NO ELIMINAR
%==================================================================
\usepackage[utf8]{inputenc}
\usepackage[T1]{fontenc}
\usepackage[spanish, es-tabla]{babel}

%==================================================================
% FORMATO GENERAL DEL DOCUMENTO
% Márgenes, gráficos, figuras y utilidades generales
% Cambiar solo si se requiere ajuste de presentación
%==================================================================
\usepackage{geometry}
% Configuración de márgenes
\geometry{
    left=2.5cm,
    right=2.5cm,
    top=2.5cm,
    bottom=2.5cm
}

\setlength{\parskip}{1\baselineskip} % Espacio entre párrafos
\usepackage{microtype}   % Mejora la justificación del texto
\usepackage{graphicx}    % Imágenes
\usepackage{float}       % Control de posición de figuras
\usepackage{enumitem}    % Listas personalizadas
\usepackage{listings}    % Código fuente
\usepackage{xcolor}      % Colores
\usepackage{tikz}        % Dibujos y marcos
\usepackage{eso-pic}     % Elementos de fondo (marco)
\usepackage{titlesec}    % Formato de títulos
\usepackage{multicol}    % Múltiples columnas
\usepackage{booktabs}    % Tablas profesionales
\usepackage{array}       % Columnas personalizadas en tablas
\usetikzlibrary{calc}    % Cálculos en TikZ (para el marco)

%==================================================================
% COLORES Y ESTILO VISUAL
% Paleta usada en encabezados, marcos y líneas decorativas
%==================================================================
\definecolor{VerdeMilitar}{RGB}{75, 83, 32}
\definecolor{termback}{rgb}{0.95,0.95,0.95}
\definecolor{codegreen}{rgb}{0,0.6,0}
\definecolor{codepurple}{rgb}{0.58,0,0.82}
\definecolor{ipcolor}{RGB}{0, 102, 204}      % Azul para IPs
\definecolor{protocolor}{RGB}{204, 0, 102}   % Magenta para protocolos
\definecolor{portcolor}{RGB}{0, 153, 76}     % Verde para puertos

%==================================================================
% CONFIGURACIÓN DE LISTINGS PARA CÓDIGO DE REDES
% Estilos predefinidos para comandos de terminal y configuraciones
%==================================================================

% Estilo base común para todos los tipos de código
\lstdefinestyle{baseStyle}{
    backgroundcolor=\color{termback},
    commentstyle=\color{codegreen}\itshape,
    numberstyle=\tiny\color{gray},
    basicstyle=\ttfamily\small,
    breaklines=true,
    frame=single,
    captionpos=b,
    numbers=left,
    numbersep=8pt,
    tabsize=4,
    showstringspaces=false
}

% Estilo para comandos Bash/Terminal
\lstdefinestyle{bashStyle}{
    style=baseStyle,
    language=bash,
    keywordstyle=\color{blue}\bfseries,
    stringstyle=\color{codepurple},
    morekeywords={ping, traceroute, ifconfig, ip, netstat, ss, nmap, tcpdump, wireshark, iptables, route, nslookup, dig, host, curl, wget},
    commentstyle=\color{codegreen}\itshape,
    morecomment=[l]{\#}
}

% Estilo para archivos de configuración de red
\lstdefinestyle{configStyle}{
    style=baseStyle,
    keywordstyle=\color{blue}\bfseries,
    stringstyle=\color{codepurple},
    commentstyle=\color{codegreen}\itshape,
    morecomment=[l]{\#}
}

% Estilo para salida de terminal (sin resaltado de sintaxis)
\lstdefinestyle{outputStyle}{
    style=baseStyle,
    keywordstyle=\color{black},
    numbers=none,
    frame=none,
    backgroundcolor=\color{black!5}
}

% Configuración por defecto (usa bashStyle)
\lstset{style=bashStyle}

%==================================================================
% COMANDOS PERSONALIZADOS PARA NETWORKING
% Facilitan el formateo consistente de elementos de red en el texto
%==================================================================

% Comando para direcciones IP (muestra en azul)
% Uso: \ip{192.168.1.1}
\newcommand{\ip}[1]{\texttt{\textcolor{ipcolor}{#1}}}

% Comando para protocolos (muestra en magenta y negrita)
% Uso: \proto{TCP}, \proto{HTTP}
\newcommand{\proto}[1]{\texttt{\textbf{\textcolor{protocolor}{#1}}}}

% Comando para puertos (muestra en verde)
% Uso: \port{80}, \port{443}
\newcommand{\port}[1]{\texttt{\textcolor{portcolor}{#1}}}

% Comando para comandos de red en línea (muestra en negrita)
% Uso: \cmd{ping}, \cmd{ifconfig}
\newcommand{\cmd}[1]{\texttt{\textbf{#1}}}

% Comando para interfaces de red (muestra en cursiva)
% Uso: \iface{eth0}, \iface{wlan0}
\newcommand{\iface}[1]{\texttt{\textit{#1}}}

%==================================================================
% HIPERVÍNCULOS
% Activa enlaces en tabla de contenido y referencias
%==================================================================
\usepackage{hyperref}
\hypersetup{
    colorlinks=true,
    linkcolor=black,
    urlcolor=blue,
    citecolor=black
}

%==================================================================
% ENCABEZADO Y PIE DE PÁGINA
% Configuración con fancyhdr
% NO MODIFICAR si no se conoce fancyhdr
%==================================================================
\usepackage{fancyhdr}
\pagestyle{fancy}
\fancyhf{}
\lhead{\footnotesize Redes de Computadoras}
\chead{\footnotesize \codigoActividad: \tituloTaller}
\rhead{\footnotesize Equipo \equipoNumero}
\cfoot{\thepage}

% Línea decorativa del encabezado
\renewcommand{\headrule}{
    {\color{VerdeMilitar}\hrule width\headwidth height 0.5pt \vskip-\headrulewidth}
}

%==================================================================
% FORMATO DE TÍTULOS
% Los títulos de sección se muestran en color VerdeMilitar
% Subsecciones y subsubsecciones en negro
%==================================================================
\titleformat{\section}
    {\large\bfseries\color{VerdeMilitar}}
    {\thesection.}{1em}{}
    
\titleformat{\subsection}
    {\normalsize\bfseries}
    {\thesubsection.}{1em}{}

\titleformat{\subsubsection}
    {\small\bfseries}
    {\thesubsubsection.}{1em}{}

%==================================================================
% TABLA DE CONTENIDO
% Cambio de nombre a español
%==================================================================
\addto\captionsspanish{\renewcommand{\contentsname}{Lista de contenido}}

%==================================================================
% DATOS DEL TRABAJO
% ESTE BLOQUE SÍ SE DEBE MODIFICAR EN CADA REPORTE
%==================================================================
\newcommand{\tituloTaller}{Nombre del Taller}
\newcommand{\unidadNumero}{X}
\newcommand{\semanaNumero}{X}
\newcommand{\codigoActividad}{TSX}
\newcommand{\equipoNumero}{05}

% --- INTEGRANTES ---
\newcommand{\integranteA}{Moya Monreal Erick Anselmo --- 1110604}
\newcommand{\integranteB}{Rodríguez Maldonado Irving Alejandro --- 1182794}
% Agrega más si es necesario

%==================================================================
% MARCO DECORATIVO DE PÁGINA
% Dibuja un marco alrededor de cada hoja
% No requiere edición
%==================================================================
\newcommand{\LogosMarco}{
    \begin{tikzpicture}[remember picture, overlay]
        \draw [line width=1.2pt, color=VerdeMilitar!90] 
            ($(current page.north west) + (0.7cm,-0.7cm)$) 
            rectangle 
            ($(current page.south east) + (-0.7cm,0.7cm)$);
        \draw [line width=0.6pt, color=VerdeMilitar!60] 
            ($(current page.north west) + (0.85cm,-0.85cm)$) 
            rectangle 
            ($(current page.south east) + (-0.85cm,0.85cm)$);
    \end{tikzpicture}
}

%==================================================================
% INICIO DEL DOCUMENTO
%==================================================================
\begin{document}
%==================================================================
% PORTADA
%==================================================================
\begin{titlepage}
    \begin{center}
        \includegraphics[width=4cm]{../recursos/logo.png}
        \hspace{1cm}
        \includegraphics[width=5cm]{../recursos/1593484343817.jpg}
        \vspace{1.5cm}

        \LARGE \textbf{Universidad Autónoma de Baja California}\\
        \Large Facultad de Ingeniería Mexicali\\
        \Large Ingeniería en Computación\\
        \vspace{0.7cm}

        {\color{VerdeMilitar}\rule{\linewidth}{0.5mm}} \\
        \large \textbf{Reporte de Actividad:}\\
        \vspace{0.1cm}
        {\Huge {\tituloTaller} }\\
        {\color{VerdeMilitar}\rule{\linewidth}{0.5mm}}

        \vspace{0.7cm}
        \begin{minipage}{0.45\textwidth}
            \textbf{Materia:}\\ Redes de Computadoras\\
            \textbf{Unidad:} \unidadNumero \quad \textbf{Semana:} \semanaNumero
        \end{minipage}
        \begin{minipage}{0.45\textwidth}
            \flushright
            \textbf{Profesor:}\\ Mtro. Emmanuel Zúñiga Torres
        \end{minipage}

        \vfill
        \textbf{EQUIPO \equipoNumero :}\\
        \integranteA \\
        \integranteB \\
        \vspace{0.5cm}
        {\color{VerdeMilitar}\rule{4cm}{0.1mm}}\\
        Mexicali, Baja California\\ \today
    \end{center}
\end{titlepage}

%==================================================================
% TABLA DE CONTENIDO
%==================================================================
\newpage
\AddToShipoutPictureBG{\LogosMarco}
\tableofcontents

%==================================================================
% SECCIÓN 1: INTRODUCCIÓN
%==================================================================
\newpage
\section{Introducción}
Aquí se presenta el contexto del taller, los recursos utilizados y el objetivo
general de la actividad. Este apartado debe permitir al lector comprender
qué se hizo y con qué propósito.

Ejemplo de uso de comandos personalizados: El servidor con IP \ip{192.168.1.100} 
escucha peticiones \proto{HTTP} en el puerto \port{80} a través de la interfaz 
\iface{eth0}.

%==================================================================
% SECCIÓN 2: DESARROLLO DEL TALLER
%==================================================================
\newpage
\section{Desarrollo de la Actividad}

%\begin{multicols}{2}
\subsection{Tema o Apartado 1}
Desarrollo del contenido principal del taller.

%\columnbreak
\subsection{Tema o Apartado 2}
Explicaciones técnicas, análisis, observaciones, etc.

%==================================================================
% EJEMPLO DE CÓDIGO BASH/TERMINAL
% Usa style=bashStyle para comandos de red
%==================================================================
\begin{lstlisting}[style=bashStyle, caption=Ejemplo de comandos de diagnóstico de red, label=lst:bash_example]
# Verificar conectividad con un servidor
ping -c 4 8.8.8.8

# Mostrar interfaces de red activas
ip addr show

# Trazar la ruta a un destino
traceroute google.com

# Ver conexiones de red activas
netstat -tuln
\end{lstlisting}

%==================================================================
% EJEMPLO DE SALIDA DE TERMINAL
% Usa style=outputStyle para mostrar salidas sin resaltado
%==================================================================
\begin{lstlisting}[style=outputStyle, caption=Salida del comando ping]
PING 8.8.8.8 (8.8.8.8) 56(84) bytes of data.
64 bytes from 8.8.8.8: icmp_seq=1 ttl=117 time=12.3 ms
64 bytes from 8.8.8.8: icmp_seq=2 ttl=117 time=11.8 ms
64 bytes from 8.8.8.8: icmp_seq=3 ttl=117 time=12.1 ms
\end{lstlisting}

%==================================================================
% EJEMPLO DE ARCHIVO DE CONFIGURACIÓN
% Usa style=configStyle para archivos de configuración
%==================================================================
\begin{lstlisting}[style=configStyle, caption=Configuración de interfaz de red]
# Configuracion estatica de eth0
auto eth0
iface eth0 inet static
    address 192.168.1.100
    netmask 255.255.255.0
    gateway 192.168.1.1
    dns-nameservers 8.8.8.8 8.8.4.4
\end{lstlisting}

%==================================================================
% EJEMPLO DE TABLA CON BOOKTABS (ESTILO PROFESIONAL)
% Usa \toprule, \midrule y \bottomrule en lugar de \hline
% NO uses líneas verticales |
%==================================================================
\subsection{Ejemplo de Tabla Profesional}
\begin{table}[H]
    \centering
    \caption{Protocolos de red más utilizados.}
    \label{tab:protocolos}
    \begin{tabular}{@{}llc@{}}
        \toprule
        \textbf{Protocolo} & \textbf{Descripción} & \textbf{Puerto} \\
        \midrule
        HTTP  & Transferencia de hipertexto & 80 \\
        HTTPS & HTTP seguro & 443 \\
        FTP   & Transferencia de archivos & 21 \\
        SSH   & Acceso remoto seguro & 22 \\
        DNS   & Sistema de nombres de dominio & 53 \\
        \bottomrule
    \end{tabular}
\end{table}

%==================================================================
% EJEMPLO DE FIGURA
%==================================================================
\begin{figure}[H]
    \centering
    \includegraphics[width=0.7\textwidth]{../taller1/google_servers.png}
    \caption{Descripción de la imagen. Imagen de prueba.}
    \label{fig:ejemplo}
\end{figure}
%\end{multicols}

%==================================================================
% SECCIÓN 3: CONCLUSIONES
%==================================================================
\newpage
\section{Conclusiones}
Conclusión general o conclusiones individuales sobre lo aprendido,
los aspectos más relevantes del taller y su importancia en el contexto
de la materia.

%==================================================================
% REFERENCIAS BIBLIOGRÁFICAS
% Podría hacerse uso de BibTeX si se desea, se automatiza
% la gestión de referencias y citas.
%==================================================================
\newpage
\addcontentsline{toc}{section}{Referencias}
\begin{thebibliography}{9}

    % Y en el texto se citan así:
    %... como se muestra en diversos centros de datos \cite{ref1}.
    \bibitem{ref1}
    Autor o Institución. \textit{Título del recurso}.
    Recuperado de: \url{https://...}

    \bibitem{ref2}
    Autor o Institución. \textit{Título del recurso}.
    Recuperado de: \url{https://...}

\end{thebibliography}

\end{document}